\documentclass{article}

\usepackage[utf8]{inputenc}
\usepackage[activeacute,spanish]{babel}
\usepackage{amsmath} %matemáticas como align
\usepackage[left=2cm,top=1cm,right=2cm,bottom=1cm,nohead,nofoot]{geometry}
\usepackage{multicol}
\usepackage{color}

\setlength{\columnseprule}{0.5pt}
\def\columnseprulecolor{\color{black}}
\setlength{\columnsep}{1cm}

\setlength{\parindent}{0pt} % Disable paragraph indentation
\pagestyle{empty} % Disable headers and footers


\begin{document}

    \centerline{\Huge{\textbf{Conjuntos}}}
    \centerline{\large{Formulario por Yael Arturo Chavoya Andalón}}

    \section{Representación}
    Si $A$ y $B$ son dos conjuntos, y $U$ es el conjunto universo, entonces:
    
    \begin{multicols}{2}
        \begin{itemize}
            \item \textbf{Unión:} $A \cup B$
            \item \textbf{Intersección:} $A \cap B$
            \item \textbf{Diferencia:} $A - B$
            \item \textbf{Diferencia simétrica:} $A \triangle B$
            \item \textbf{Complemento:} $A^C$ ó $A'$
            \item \textbf{Subconjunto:} $A \subset B$
            \item \textbf{Cardinalidad:} $|A|$ ó $\#A$ ó $Card |A|$
            \item \textbf{Potencia:} $P(A)$
        \end{itemize}
    \end{multicols}

    \section{Leyendas del Álgebra de conjuntos}

    \begin{multicols}{2}

        \textbf{Leyes idempotentes}
        \begin{subequations}
            \begin{align}
                A \cup A = A\\
                A \cap A = A
            \end{align}
        \end{subequations}

        \textbf{Leyes asociativas}
        \begin{subequations}
            \begin{align}
                (A \cup B) \cup C &= A \cup (B \cup C)\\
                (A \cap B) \cap C &= A \cap (B \cap C)
            \end{align}
        \end{subequations}

        \textbf{Leyes conmutativas}
        \begin{subequations}
            \begin{align}
                A \cup B &= B \cup A\\
                A \cap B &= B \cap B
            \end{align}
        \end{subequations}

        \textbf{Leyes distributivas}
        \begin{subequations}
            \begin{align}
                A \cup (B \cap C) &= (A \cup B) \cap (A \cup C)\\
                A \cap (B \cup C) &= (A \cap B) \cup (A \cap C)
            \end{align}
        \end{subequations}

        \textbf{Leyes de identidad y absorción}
        \begin{subequations}
            \begin{align}
                A \cup \phi &= A\\
                A \cap U &= A \\
                A \cup U &= U\\
                A \cap \phi &= \phi
            \end{align}
        \end{subequations}

        \textbf{Ley involutiva}
        \begin{align}
            (A^C)^C &= A
        \end{align}

        \textbf{Leyes del complementario}
        \begin{subequations}
            \begin{align}
                A \cup A^C &= U\\
                A \cap A^C &= \phi \\
                U^C &= \phi\\
                \phi^C &= U
            \end{align}
        \end{subequations}
        
    \end{multicols}

    \section{Principios de Conteo}

    \subsection{Con conjuntos disjuntos}

    Si $A$ y $B$ son dos conjuntos disjuntos, entonces:

    \begin{equation}
        | A \cup B | = | A | + | B |
    \end{equation}

    \subsection{Con conjuntos no disjuntos}

    Si $A$ y $B$ son dos conjuntos cualesquiera, entonces:

    \begin{equation}
        | A \cup B | = | A | + | B | - | A \cap B |
    \end{equation}

    Observe que la última relación también se cumple cuando $A$ y $B$ son disjuntos, es decir, $A \cap B = \phi$, ya que $|\phi| = 0$.

    \section{Propiedades de la diferencia de conjuntos}

    Las propiedades siguientes, llamadas \textit{leyes de Morgan}, se cumplen para conjuntos $A$ y $B$ que son subconjuntos del conjunto universal $U$.

    \centering{\textbf{Leyes de Morgan}}
    \begin{subequations}
        \begin{align}
            (A \cup B)^C &= A^C \cap B^C\\
            (A \cap B)^C &= A^C \cup B^C
        \end{align}
    \end{subequations}

    


\end{document}